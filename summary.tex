\chapter{Summary}

Having a clear image in your mind for what you want out of life is important;
otherwise, we'd never actually get what we want, because we don't know what we're
chasing. It's also very easy to get distracted from the image you've envisioned,
because there's nothing concrete holding you accountable for the steps you take
to get to that end result --- as a result, we find ourselves sidetracked with
meaningless nonsense when we actually really wanted to spend our years doing
something productive.

This document aims to help me figure out exactly what I want to be, where I want
to be, and how I'm going to get there. There are many things to consider ---
physical possessions (i.e., finances, vehicle, etc.), character traits,
relationships, and even personal motivations. It's hard to change your behavior
and character, but where there's a will, there's a way, and I don't want to
see myself missing out on more opportunities in my life --- not opportunities to
party and have a good time, but to let it go to waste.

GySgt Joiner advised us to take the opportunity to direct ourselves, and that's
some of the wisest advice I think I've heard. He is a very realistic and intelligent
man; I hope to have soaked up some of his appreciation for the opportunities we
have now.

\section{``What type of Marine do you want to be?"}

Those words never rang more true. He asked some very important questions that I
was lucky enough to listen to his \textit{previous} advise, in writing down.
The following were mentioned, and in this paper I'd like to add more structure
and rigor to their values:

\begin{itemize}
\item{List your goals for your USMC tenure}
\item{What type of Marine do you want to be?}
\item{Control your impression}
\item{SMART goals - short term and long term}
\item{Define your path to better yourself}
\item{What kind of Corps do you want to serve?}
\item{What do you want to accomplish?}
\item{Put down the dream-killers --- what are they?}
\item{A marathon is a product of individual steps}
\item{Be as productive as possible --- get immersed in your job \textit{first}}
\item{If you waste your potential, you let \textbf{yourself} down}
\item{What have you already done to get to your goals?}
\item{What will you sacrifice? And what is practical?}
\item{How can you work with others to help you get to your goals?}
\item{How will you manage your time requirements?}
\item{What are good starter goals, to get familiar with the process?}
\item{How will you pace yourself and not sacrifice the wrong things?}
\item{Earn your reputation}
\item{Pay your dues}
\item{What books could you read to improve this process?}
\item{What books should you read regardless?}
\item{You either have ``it", or you don't. Set the expectation that others should strive to have}
\item{What long term goals have value that's difficult to measure?}
\item{How will you revise your goals regularly?}
\item{Your decisions \textbf{are} your dues}
\item{Accomplish what you're told}
\item{Ask for permission and sympathy when appropriate}
\item{You do not need your hand held}
\item{Take an English 1 \& 2 course (eventually)}
\item{Planning is \textit{key}}
\end{itemize}

The values of the above statements will be exercised below. If uninterested, please skip to the
next chapter.

\subsection{List your goals for your USMC tenure}

This is very important. Most people think of being a Marine as just a ``job" --- often,
I reflect on how my NCOs would remark as ``Being a Marine is the easiest job
you'll ever have! All you have to do is show up on time, shave your face, get
a haircut, and do as your told!", while grilling us on some negligence or another.

At any rate, being a Marine is more than just a job; it's a \textit{tenure} --- if
you don't do something stupid, or get hurt, then it's a job that lasts \textit{at least}
four years. I don't know about you, but that's the longest job I've ever had.
A lot can be done in that amount of time, and it's important to think about how
things will change for you; what trends are stronger than others. However, if you
are influential enough, you can guide that trend into a different direction.
At any rate, there is a lot that can be accomplished in four years.

\subsection{What type of Marine do you want to be?}

This is also a tricky question, because there's so many stereotypes we learn about
as junior Marines; the shitbags, the unicorns, ``water walkers", hardasses,
badasses, salty bastards, etc.

When you get caught on a good day, what impression do you want to leave? What
about a bad day? What about an average day? What about during a firefight?
What about at the gym, or during PT? what about when you've been caught with
your pants down?

\subsection{Control your Impression}

This deals with a separate paper's topic as well, but controlling your impression
is important --- staying tight with your chain of command; knowing what risks
they take entrusting you with certain tasks, and knowing what they care about
is very important. Furthermore, being overly prepared in cause you're put on
a promotion board (or, God forbid, a court) can be very useful --- having
a jacket of impressionable material.

\subsection{SMART Goals; Short Term and Long Term}

Some of this will be taken from \href{https://corporatefinanceinstitute.com/resources/knowledge/other/smart-goal/}{this website}.

Goals are part of every aspect of business/life and provide a sense of direction,
motivation, a clear focus, and clarify importance. By setting goals for
yourself, you are providing yourself with a target to aim for. A SMART goal
is used to help guide goal setting. SMART is an acronym that stands for Specific,
Measurable, Achievable, Realistic, and Timely. Therefore, a SMART goal incorporates
all of these criteria to help focus your efforts and increase the chances of
achieving that goal.

SMART goals are:

\begin{itemize}
\item{Specific: Well defined, clear, and unambiguous}
\item{Measurable: With specific criteria that measure your progress towards the accomplishment of the goal}
\item{Achievable: Attainable and not impossible to achieve}
\item{Realistic: Within reach, realistic, and relevant to your life purpose}
\item{Timely: With a clearly defined timeline, including a starting date and a target date. The purpose is to create urgency.}
\end{itemize}

Having time-indexed, proactive goals will be crucial to rapid accomplishment.

\subsection{Define your path to better yourself}

This is an overarching statement about the importance of goal setting ---
in my dad's loving words, ``you are the master of your disaster",
and ``you're clear for takeoff, it's your life".

Following other people's footsteps does have it's values, but it's easy to
get misguided or lose sight of your own initiatives. Setting your rubric for
your life plan will help you abide to your own expectations.

\subsection{What kind of Corps do you want to serve?}

It's a hard question to answer as a junior Marine, but still very important ---
what behavior will you enable? What is permissible, and what isn't? Who do
you want to support, and what policies will you rally behind? Will you
be outspoken about your opinions?

\subsection{What do you want to accomplish?}

At the end of the day, years from now, I'll want to feel proud of myself.
What will I want to be proud of? What will I want other people to be proud
of? What will give me what I need? And what will get me further to long term
goals? What do I want to be known for?

\subsection{Put down the dream-killers --- what are they?}

This is a hard one to address, because it's closely related to enjoyment.
Everyone ``needs" to ``relax" sometimes --- take your mind off work with
a dumb distraction. But in reality, maybe there might be a way to have
fun while being productive. Some things might actually help, though,
like a useful distraction, or a change of pace that keeps you aware
and not drained, yet keeps your mind versatile.

\subsection{A marathon is a product of individual steps}

And each step must be measured --- well, maybe not every step, but
knowing where the next one should be is damn important, else you might trip!

\subsection{Be as productive as possible --- get immersed in your job \textit{first}}

This is a hard one, especially as a new join for my permanent duty station.
However, it's extremely important; I'll need to be more productive, sharper,
and most trustworthy if I want to get promoted.

\subsection{If you waste your potential, you let \textbf{yourself} down}

This breaks my heart, because I know it's true, about my lackadaisical wastefulness
of my past. I have been driven for a long time now, however, so I shouldn't feel so
guilty. But, maybe it's the lasting effects of being reckless too.

It's very important to stay true to yourself; not necessarily the ``self" that
wants to get drunk and see what girl he \textit{actually} wants, but the self
that regrets decisions, and can give wise advice to a youngster about both their
mistakes and pride.

\subsection{What have you already done to get to your goals?}

Don't short-sell yourself, you've lived a long life so far. It's important
to know what you've already done, too, so you can better estimate how productive you
can actually be when you really get to work.

\subsection{What will you sacrifice? And what is practical?}

Sacrifice is the reality of exchanging time and energy, and what we actually want.
I would like to be the kind of person that can rationally make a decision for
sacrificing comfort, without ``feeling" it, so to speak.

\subsection{How can you work with others to help you get to your goals?}

I don't mean this by getting others to do your work, but I'm also not
saying you have to do everything alone. Maybe there's another individual who's
goals coincide with yours, and you could symbiotically benefit from each other.
Likewise, how can you get your chain of command involved with your goals, not
intimately, but just aware enough about something they approve of, so you don't
have to make the decision to sacrifice necessities to accomplish them.

\subsection{How will you manage your time requirements?}

Schedules are very useful, but sometimes issues (like field ops) come out of
nowhere and put a stick in the spokes of your routine. Being agile and aware
of how to change your footing will be important to staying productive in a
changing environment.

\subsection{What are good starter goals, to get familiar with the process?}

Low hanging fruit are important to harvest; it can improve morale and give
a better sense of judgment for other goals' which may be more difficult to
attain. However, when starting a new routine, it will be important to
\textit{go slow}.

\subsection{How will you pace yourself and not sacrifice the wrong things?}

Ask for advice, or if you're not comfortable with that, ease your way into the
change of process that you'll face when going head-on.

\subsection{Earn your reputation}

You will not work alone, and you are not ``safe" from the same issues everyone
else will face. It's important to make a robust image of yourself; that you can
handle whatever comes your way, because you meet (and ideally exceed) your
expectations.

\subsection{Pay your dues}

There will be a lot of work ahead of you, and as a new join, you are in a debt;
paying back that debt is your first priority.

\subsection{What books could you read to improve this process?}

Are there any good self-help books that would give you valuable perspective?
Like ``how to make friends and influence people"?

\subsection{What books should you read regardless?}

The Commandant's reading list is out there, and we are tasked with reading 3
per year, and writing reports on them.

\subsection{You either have ``it", or you don't. Set the expectation that others should strive to have}

Go-getters are what make the world turn. However, it's a constant grind, and
easy to let go. You should make an image for those who need us, that sets us apart
from everyone else.

\subsection{What long term goals have value that's difficult to measure?}

For instance, after 10 years in the USMC, your GI bill can go to your
dependents. As someone without dependents, that's a very interesting possibility,
because I don't know what I would be missing.

\subsection{How will you revise your goals regularly?}

Set a weekly, monthly, quarterly schedule for revising your schedule, and goal iteration.

\subsection{Your decisions \textbf{are} your dues}

A man is as valuable as his word, especially if he involves other people
in making those opportunities available.

\subsection{Accomplish what you're told}

And maintain communication for what isn't up-to-par. Surprises aren't very nice
for someone who manages expectations.

\subsection{Ask for permission and sympathy when appropriate}

When growing, you'll need some help getting flexible. Don't be afraid to be humble,
and ask for advice.

\subsection{You do not need your hand held}

Do everything that \textit{can} be done independently, without overstepping your
boundaries. Preparation and planning are the keys to success.

\subsection{Take an English 1 \& 2 course (eventually)}

This will help you maintain good, impressive communication with important people.

\subsection{Planning is \textit{key}}

This is what will set you apart from everyone else, who will waste their time
doing dumb shit. Making a plan, and sticking to that plan, will help you grow
as a person, and as a Marine.
